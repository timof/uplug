\documentclass[11pt]{article}

\usepackage{german}
\usepackage{times}

% \fontsize{11pt}{8pt}\selectfont
\def\normalfont{\sffamily}

\parindent0pt
\topmargin-15mm
\headsep0pt
\oddsidemargin-10mm
\leftskip0pt
\textwidth180mm
\textheight260mm

\rightskip0pt plus1ex minus1em
\hyphenpenalty6000
\spaceskip0.33em plus0.2em minus0.03em

\def\items#1{{%
  \itcounter0%
  \begin{itemize}
  #1
  \end{itemize}
}}

\newcount\itcounter
\itcounter0
\let\it\item%
\def\item{
  \advance\itcounter1%
  \it[(\the\itcounter)]
}
 
\def\paragraf#1{
  \advance\itcounter1%
  \par\medskip
  {\large\bfseries \S\,\the\itcounter\quad#1}
  \par\smallskip
}

\begin{document}
\normalfont
\fontsize{11pt}{8pt}\selectfont

\centerline{\Large\bfseries  Satzung der Universit{\"a}t Potsdam --- LinuxUserGroup}
\par
\bigskip

\paragraf{Name, Sitz und Gesch{\"a}ftsjahr}
\medskip

Der Verein f{\"u}hrt den Namen \glqq Universit{\"a}t Potsdam - LinuxUserGroup\grqq{} (upLUG).
Das Gesch{\"a}ftsjahr ist das Kalenderjahr.
Sitz des Vereins ist Potsdam.
Eintragung beim Amtsgericht Potsdam wird angestrebt.
%
% "gemeinnuetzig" gehoert hier nicht hin - damit hat das AG nix zu tun
%
\bigskip

\paragraf{Zweck des Vereins}

\items{
  \item
    Der Verein verfolgt ausschlie{\ss}lich und unmittelbar gemeinn{\"u}tzige Zwecke
    im Sinne des Abschnitts \glqq Steuerbeg{\"u}nstigte Zwecke\grqq{} der Abgabenordnung.
  \item
     Zweck des Vereins ist die F{\"o}rderung von Bildung, Verst{\"a}ndnis und
     Akzeptanz sowie die Unter\-st{\"u}tzung von Anwendern, insbesondere im privaten Bereich
     und im Bereich der Schulen und Hochschulen, in Zusammenhang mit der Datenverarbeitung
     unter spezieller Ber{\"u}cksichtigung des frei ver\-f{\"u}g\-baren Betriebssystems Linux.
   \item
     Der Satzungszweck wird insbesondere verwirklicht durch
      \items{
        \item
          {\"o}ffentliche und eintrittsfreie
          Vortr{\"a}ge, Workshops und andere Veranstaltungen zu Linux und anderer
          freier Software;
        \item
          unentgeltliche Unterst{\"u}tzung und Beratung von Studierenden und anderen Personen
          in Fragen der Inbetriebnahme und Benutzung von Linux und anderer freier Software.
          % "Hilfe fuer Studierende" ist auch so ein Schluesselbegriff fuer Gemeinnuetzigkeit
      }
   \item
     Der Verein ist selbstlos t{\"a}tig und verfolgt nicht in erster Linie eigenwirtschaftliche Zwecke.
     \items{
       \item
         Mittel des Vereins d{\"u}rfen nur f{\"u}r satzungsgem{\"a}{\ss}e Zwecke verwendet werden.
       \item
         Die Mitglieder des Vereins erhalten in ihrer Eigenschaft als Mitglied
         keine Zuwendungen aus Mitteln des Vereins. Die T{\"a}tigkeit des Vorstands f{\"u}r den
         Verein wird nicht verg{\"u}tet.
       \item
         Keine Person darf durch Ausgaben, die dem Zweck der K{\"o}rperschaft fremd sind, oder durch
         unverh{\"a}ltnism{\"a}{\ss}ig hohe Verg{\"u}tungen beg{\"u}nstigt werden.
       \item
         Nachgewiesene Auslagen im Rahmen ehrenamtlicher T{\"a}tigkeit f{\"u}r den Verein
         k{\"o}nnen erstattet werden.
     }
}

\paragraf{Mitgliedschaft}

\items{
   \item Der Verein hat Mitglieder und F{\"o}rdermitglieder. Mitglied oder
   F{\"o}rdermitglied kann jede nat\"urliche oder juristische Person werden.
   \item {\"U}ber die Aufnahme entscheidet nach schriftlichem Antrag der Vorstand.
       {\"U}ber die Entscheidung wird schriftlich an die dem Verein bekannten Adresse informiert.
       Im Falle der Ablehnung kann Berufung zur n{\"a}chsten
       ordentlichen Mitgliederversammlung eingelegt werden.
   \item Der Austritt eines Mitgliedes erfolgt durch schriftliche Erkl{\"a}rung gegen{\"u}ber
       dem Vorstand. Er wird mit Ablauf des Folgemonats wirksam.
   \item Ein Mitglied kann aus wichtigem Grund ausgeschlossen werden. Ein
       wichtiger Grund liegt vor, wenn sein Verhalten in grober Weise gegen die
       Interessen des Vereins oder die Pflichten als Mitglied verst{\"o}{\ss}t.
       {\"U}ber den Ausschluss entscheidet die Mitgliederversammlung;
       die Einladung zu dieser Mitgliederversammlung muss den Ausschluss des
       Mitgliedes als Tagesordnungspunkt enthalten.
       {\"U}ber den Ausschluss kann die Mitgliederversammlung auch in
       Abwesenheit des betroffenen Mitgliedes beschlie{\ss}en, wenn
       dem auszuschlie{\ss}enden Mitglied mit der Ladung zur Mitgliederversammlung
       dieser Sachverhalt und der Ausschlie{\ss}ungsgrund
       % schriftlich an die dem Verein bekannte Adresse
       % ^ ist allmemein fuer _alle_ einladungen zur MV so geregelt!
       mitgeteilt wurde.
}

\paragraf{Rechte und Pflichten der Mitglieder}

\items{
   \item Die Mitglieder sind berechtigt, an allen Veranstaltungen des Vereins
       teilzunehmen und die Einrichtungen des Vereins zu beanspruchen. Das
       Stimmrecht in der Mitgliederversammlung sowie das aktive und passive
       Wahlrecht steht allen Mitgliedern, nicht jedoch F{\"o}rdermitgliedern, zu.
       In der Mitgliederversammlung kann das Stimmrecht nur pers{\"o}nlich ausge{\"u}bt werden.
   \item Die Mitglieder sind verpflichtet, die Interessen des Vereins nach Kr{\"a}ften zu
       f{\"o}rdern und alles zu unterlassen, wodurch das Ansehen und der Zweck des
       Vereins Schaden erleiden k{\"o}nnte. Sie haben die Vereinssatzung und die
       Beschl{\"u}sse der Vereinsorgane zu beachten.
   \item Die Mitgliederversammlung kann eine Beitragsordnung erlassen, die H{\"o}he und
       Turnus von Mitgliedsbeitr{\"a}gen regelt.
   \item Die Mitgliederversammlung kann eine Regelung beschlie{\ss}en, die Mitglieder
       zur Leistung ehrenamtlicher T{\"a}tigkeit f{\"u}r den Verein verpflichtet.
   \item Die Mitglieder sind verpflichtet, Adress{\"a}nderungen dem Verein schriftlich mitzuteilen.
}

\paragraf{Vereinsorgane}
\medskip

Organe des Vereins sind die Mitgliederversammlung und der Vorstand.
\bigskip

\paragraf{Vorstand}

\items{
   \item
     Der Vorstand besteht aus mindestens zwei Personen.
     Die Mitgliederversammlung kann eine h{\"o}here Anzahl von Vorstandsmitgliedern beschlie{\ss}en.
   \item
     Der Vorstand wird f\"ur die Dauer bis zur n\"achsten ordentlichen Mitgliederversammlung
     gew{\"a}hlt. Wiederwahl ist zul{\"a}ssig, ebenso Abwahl und Neuwahl durch eine au{\ss}erordentliche
     Mitgliederversammlung.
   \item
     Der Verein wird nach \S\,26 BGB gerichtlich und au{\ss}ergerichtlich durch je zwei
     Vorstandsmitglieder gemeinschaftlich vertreten.
   \item
     Der Vorstand ist f{\"u}r alle Angelegenheiten des Vereins zust{\"a}ndig, soweit sie nicht
     durch die Satzung der Mitgliederversammlung zugewiesen sind, insbesondere
     \items{
       \item Vorbereitung und Einberufung der Mitgliederversammlung;
       \item F{\"u}hrung der laufenden Vereinsgesch{\"a}fte und der Mitgliederliste;
       \item Vertretung des Vereins;
       \item Ausf{\"u}hrung von Beschl{\"u}ssen der Mitgliederversammlung.
     }
   \item
     Der Vorstand ist beauftragt, rein redaktionelle Satzungs{\"a}nderungen auf Verlangen
     des Amtsgerichts oder des Finanzamtes oder die im Zusammenhang mit einer Vereinsregistereintragung
     notwendig werden, auch ohne Beschluss der Mitgliederversammlung durchzuf{\"u}hren.
   \item
     Die Haftung der Vorstandsmitglieder gegen{\"u}ber dem Verein wird auf Vorsatz und grobe
     Fahrl{\"a}ssigkeit %und der H\"ohe nach auf das Vereinsverm\"ogen (siehe BGB §31a)
     beschr{\"a}nkt.
}

\paragraf{Mitgliederversammlung}

\items{
   \item Oberstes Vereinsorgan ist die Mitgliederversammlung.
   \item Die Mitgliederversammlung ist insbesondere zust{\"a}ndig f{\"u}r
     \items{
       \item die Entlastung des Vorstands nach Entgegennahme des Jahresberichts und der
                Stellungnahme der Rechnungspr{\"u}fer;
       \item die Genehmigung des Haushaltsplanes;
       \item die Wahl des Vorstands und zweier Rechnungspr{\"u}fer;
       \item Beschlussfassung {\"u}ber Satzungs{\"a}nderungen, die Beitragsordnung, Ausschluss von
                Mitgliedern und die Aufl{\"o}sung des Vereins.
     }
   \item Die ordentliche Mitgliederversammlung findet einmal j{\"a}hrlich statt.
   \item Eine au{\ss}erordentliche Mitgliederversammlung wird anberaumt
     \items{
       \item auf Beschluss des Vorstands oder der Mitgliederversammlung;
       \item auf Antrag eines Rechnungspr{\"u}fers oder eines
                Zehntels der Mitglieder; in diesem Fall ist die Versammlung innerhalb von
                vier Wochen nach Antragseingang durchzuf{\"u}hren.
     }
   \item Zu jeder Mitgliederversammlung sind alle Mitglieder mindestens zwei Wochen vor
       dem Termin schriftlich einzuladen.
       % fuer eine LUG gilt "email" als "schriftlich": siehe (2) in http://www.gesetze-im-internet.de/bgb/__127.html
       % (schlimmstenfalls muessen wir dann auf ausdruecklichen wunsch die email ausdrucken und unterschreiben)
       Die Einladung erfolgt durch ein Mitglied des Vorstands an die letzte dem Verein bekannte Adresse.
   \item
       Die Einladung zur Mitgliederversammlung erfolgt unter Angabe der Tagesordnung.
       Beschlussfassung ist nur m{\"o}glich {\"u}ber Gegenst{\"a}nde, die in der Tagesordnung
       genannt sind. Davon abweichend ist ein Beschluss zur Einberufung einer
       au{\ss}erordentlichen Mitgliederversammlung jederzeit m{\"o}glich.
       Zur Beschlussfassung {\"u}ber eine {\"A}nderung der Satzung muss die vorgeschlagene
       Neufassung der Einladung im Wortlaut beif{\"u}gt werden.
    \item 
       Gegenst{\"a}nde sind in die Tagesordnung aufzunehmen, soweit dies unter Einhaltung der
       Ladefrist m{\"o}glich ist,
      \items{
        \item auf Antrag eines Vorstandsmitglieds oder eines Rechnungspr{\"u}fers,
        \item auf Antrag eines Zehntels der Mitglieder,
        \item auf Antrag eines Drittels der zur Mitgliederversammlung erschienenen Mitglieder.
       }
   \item
       Die Mitgliederversammlung ist {\"o}ffentlich. Der Vorstand bestimmt zu Beginn der Mitglieder\-ver\-sammlung
       im Einvernehmen mit der Mitgliederversammlung einen Versammlungsleiter und einen Schriftf{\"u}hrer.
   \item 
         Jede ordnungsgem{\"a}{\ss} einberufene Mitgliederversammlung ist ohne R{\"u}cksicht auf
         die Anzahl der erschienenen Mitglieder beschlussf{\"a}hig.
         Davon abweichend erfordern Beschl{\"u}sse {\"u}ber Satzungs\-{\"a}nderungen oder
         die Aufl{\"o}sung des Vereins die
         Anwesenheit der H{\"a}lfte aller Mitglieder. Liegt insoweit Beschlussunf{\"a}higkeit
         vor, kann die Mitgliederversammlung die Einberufung eine au{\ss}erordentlichen
         Mitgliederversammlung beschlie{\ss}en, die ohne R{\"u}cksicht auf die Zahl der Erschienenen zum
         Beschluss {\"u}ber die neue Satzung oder die Aufl{\"o}sung berechtigt ist.
   \item
     Die Mitgliederversammlung beschlie{\ss}t
     \items{
       \item {\"u}ber die Aufl{\"o}sung des Vereins oder {\"A}nderung des Vereinszwecks mit  einer
              Mehrheit von 3/4 der anwesenden Mitglieder;
       \item {\"u}ber sonstige Satzungs{\"a}nderungen und Ausschluss von Mitgliedern mit einer Mehrheit von
                2/3 der anwesenden Mitglieder;
       \item in allen anderen F{\"a}llen mit einfacher Mehrheit.
     }
   \item Die gefassten Beschl{\"u}sse werden schriftlich niedergelegt und vom
        Versammlungsleiter und vom Schriftf{\"u}hrer unterzeichnet. Das Protokoll enth{\"a}lt
        Ort und Zeit der Versammlung sowie anwesende Mitglieder und Abstimmungsergebnisse.
}



\paragraf{Hochschulgruppe}

\items{
  \item
    Diejenigen Mitglieder des Vereins, die zugleich Mitglieder Brandenburger Hochschulen oder
    Fachhochschulen sind, bilden die \glqq upLUG Hochschulgruppe\grqq{} (HG).
  \item
    Mitgliederversammlungen des Vereins sind zugleich Mitgliederversammlungen der HG.
    {\"U}ber Angelegenheiten, die nur die HG betreffen, sind nur die Mitglieder der HG
    stimmberechtigt.
  \item
    Diejenigen Mitglieder des Vereinsvorstandes, die zugleich
    Mitglieder der HG sind, bilden den Vorstand der HG, sofern die HG nichts anderes beschlie{\ss}t.
}

\paragraf{Aufl{\"o}sung des Vereins}

\items{
   \item
     Sofern die Mitgliederversammlung nichts anderes beschlie{\ss}t, wird im Falle der Aufl{\"o}sung
     der letzte Vereinsvorstand mit der Liquidation beauftragt. Je zwei Liquidatoren sind
     gemeinsam vertretungsberechtigt.
   \item
     Im Falle der Aufl{\"o}sung oder Aufhebung des Vereins oder bei Wegfall steuerbeg{\"u}nstigter
     Zwecke f{\"a}llt das nach Begleichung aller Verbindlichkeiten verbleibende Verm{\"o}gen
     an die Studierendenschaft der Universit{\"a}t Potsdam vertreten durch
     den Allgemeinen Studierendenausschuss der Universit{\"a}t Potsdam.
     Die genannte Institution hat das Verm{\"o}gen unmittelbar und ausschlie{\ss}lich f{\"u}r
     gemeinn{\"u}tzige oder mildt{\"a}tige Zwecke zu verwenden.
}

\par\bigskip
Potsdam, den 29.\,07.\,2012

\end{document}


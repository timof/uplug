\documentclass[11pt]{article}

\usepackage{german}
\usepackage{times}

\sffamily
\fontsize{12pt}{10pt}\selectfont

\parindent0pt
\topmargin-10mm
\headsep0pt
\oddsidemargin-10mm
\leftskip0pt
\textwidth180mm
\textheight260mm

\rightskip0pt plus1em
\hyphenpenalty2000

\def\items#1{
  \begin{itemize}
  #1
  \end{itemize}
}

\newcount\itcounter
\newcount\paragrafcounter
\itcounter0
\let\it\item%
\def\item{
  \advance\itcounter1%
  \it[(\the\itcounter)]
}
 
\def\paragraf#1#2{
  \par\bigskip
  {\large\bfseries \S\,#1\quad#2}
  \par\medskip
}

\begin{document}
\sffamily
\fontsize{11pt}{8pt}\selectfont

\centerline{\Large\bfseries  Satzung der Universit{\"a}t Potsdam --- LinuxUserGroup}
\par
\bigskip

\paragraf1{Name, Sitz und Gesch{\"a}ftsjahr}

Der Verein f{\"u}hrt den Namen \glqq Universit{\"a}t Potsdam - LinuxUserGroup\grqq (upLUG)
Der Verein hat seinen Sitz in Potsdam. Das Gesch{\"a}ftsjahr ist das Kalenderjahr.
Eintragung beim Amtsgericht Potsdam wird angestrebt.
%
% "gemeinnuetzig" gehoert hier nicht hin - damit hat das AG nix zu tun
%
% Adresse muss hier nicht stehen, also: besser weglassen!

\paragraf2{Zweck des Vereins}

\items{
  \item
    Der Verein verfolgt ausschlie{\ss}lich und unmittelbar gemeinn{\"u}tzige Zwecke
    im Sinne des Abschnitts \glqq Steuerbeg{\"u}nstigte Zwecke\grqq{} der Abgabenordnung.
  \item
     Zweck des Vereins ist die F{\"o}rderung von Bildung, Verst{\"a}ndnis und
     Akzeptanz sowie die Unterst{\"u}tzung von Anwendern im Bereich der Datenverarbeitung
     unter spezieller
     Ber{\"u}cksichtigung des frei verf{\"u}gbaren Betriebssystems Linux.
   \item
      Der Satzungszweck wird insbesondere verwirklicht durch
      \items{
        \it[(a)]
          {\"o}ffentliche und eintrittsfreie
          % Schluesselwoerter fuer Gemeinnuetzigkeit!
          Veranstaltungen, Vortr{\"a}ge und Workshops zu Linux und anderer
          freier Software;
        \it[(b)]
          unentgeltliche Unterst{\"u}tzung und Beratung von Studierenden und anderen Personen
          in Fragen der Inbetriebnahme und Benutzung von Linux und anderer freier Software.
          % "Hilfe fuer Studierende" ist auch so ein Schluesselbegriff fuer Gemeinnuetzigkeit
      }
   \item
     Der Verein ist selbstlos t{\"a}tig und verfolgt keine eigenwirtschaftlichen Zwecke.
     \items{
       \it[(a)]
         Mittel des Vereins d{\"u}rfen nur f{\"u}r satzungsgem{\"a}{\ss}e Zwecke verwendet werden.
       \it[(b)]
         Die Mitglieder des Vereins erhalten in ihrer Eigenschaft als Mitglied
         keine Zuwendungen aus Mitteln des Vereins. Die T{\"a}tigkeit des Vorstands f{\"u}r den
         Verein wird nicht verg{\"u}tet.
       \it[(c)]
         Nachgewiesene Auslagen im Rahmen ehrenamtlicher T{\"a}tigkeit f{\"u}r den Verein
         k{\"o}nnen erstattet werden.
     }
}

\paragraf3{Mitgliedschaft}

\items{
   \item Der Verein hat Mitglieder und F{\"o}rdermitglieder. Mitglied oder
   F{\"o}rdermitglied kann jede nat\"urliche Person werden.
   \item {\"U}ber die Aufnahme entscheidet nach schriftlichem Antrag der Vorstand.
       Erfolgt eine Ablehnung, kann innerhalb eines Monats Berufung zur n{\"a}chsten
       ordentlichen Mitgliederversammlung eingelegt werden.
   \item Der Austritt eines Mitgliedes erfolgt durch schriftliche Erkl{\"a}rung gegen{\"u}ber
       dem Vorstand. Er wird einen Monat, nachdem er erkl{\"a}rt wird, wirksam.
   \item Ein Mitglied kann aus wichtigem Grund ausgeschlossen werden. Ein
       wichtiger Grund liegt vor, wenn sein Verhalten in grober Weise gegen die
       Interessen des Vereins verst{\"o}{\ss}t.
       �ber den Ausschluss entscheidet die Mitgliederversammlung mit
       Zweidrittelmehrheit. Die Einladung zu dieser Mitgliederversammlung muss unter
       Wahrung der satzungsgem{\"a}{\ss}en Frist erfolgen und den Ausschluss des
       Mitgliedes als Tagesordnungspunkt enthalten.
       {\"U}ber den Ausschluss kann durch die Mitgliederversammlung auch in
       Abwesenheit des betroffenen Mitgliedes wirksam abgestimmt werden, wenn
       dem auszuschlie{\ss}enden Mitglied mit der Ladung zur Mitgliederversammlung
       schriftlich mitgeteilt wurde, dass und aus welchen Gr{\"u}nden {\"u}ber seinen
       Ausschluss in der Mitgliederversammlung abgestimmt werden soll und dass
       die Abstimmung auch in seiner Abwesenheit erfolgen kann.
}

\paragraf4{Rechte und Pflichten der Mitglieder}

\items{
   \item Die Mitglieder sind berechtigt, an allen Veranstaltungen des Vereins
       teilzunehmen und die Einrichtungen des Vereins zu beanspruchen. Das
       Stimmrecht in der Mitgliederversammlung sowie das aktive und passive
       Wahlrecht steht allen Mitgliedern zu.
       In der Mitgliederversammlung kann das Stimmrecht nur pers{\"o}nlich ausge{\"u}bt werden.
   \item Die Mitglieder sind verpflichtet, die Interessen des Vereins nach Kr{\"a}ften zu
       f{\"o}rdern und alles zu unterlassen, wodurch das Ansehen und der Zweck des
       Vereins Abbruch erleiden k{\"o}nnte. Sie haben die Vereinssatzung und die
       Beschl{\"u}sse der Vereinsorgane zu beachten.
}

\paragraf5{Vereinsorgane}

Organe des Vereins sind die Mitgliederversammlung und der Vorstand.

\paragraf6{Vorstand}

\items{
   \item
     Der Vorstand besteht aus mindestens zwei Personen, darunter ein Vorsitzender und ein
     Stellvertretender Vorsitzender. Die Mitgliederversammlung
     kann eine h{\"o}here Anzahl von Vorstandsmitgliedern beschliessen.
   \item
     Der Vorstand wird f\"ur die Dauer bis zur n\"achsten ordentlichen Mitgliederversammlung
     gew{\"a}hlt. Wiederwahl ist zul{\"a}ssig
   \item
     Der Verein wird nach \S\,26 BGB gerichtlich und au{\ss}ergerichtlich durch je zwei
     Vorstandsmitglieder gemeinschaftlich vertreten.
   \item Scheidet ein Vorstandsmitglied w{\"a}hrend der Wahlperiode aus, werden alle
       Mitglieder unter Einhaltung der satzungsgem{\"a}{\ss}en Frist schriftlich zu einer
       au{\ss}erordentlichen Mitgliederversammlung geladen, auf der ein neues
       Vorstandsmitglied gew{\"a}hlt wird.
}

\paragraf7{Mitgliederversammlung}

\items{
   \item Oberstes Vereinsorgan ist die Mitgliederversammlung.
   \item Die Mitgliederversammlung ist insbesondere zust{\"a}ndig f{\"u}r
     \items{
       \it[(a)] die Entlastung des Vorstands nach Entgegennahme des Jahresberichts und der
                Stellungnahme der Rechnungspr{\"u}fer;
       \it[(b)] die Genehmigung des Haushaltsplanes;
       \it[(c)] die Wahl des Vorstands;
       \it[(d)] die Wahl von zwei Rechnungspr{\"u}fern;
       \it[(e)] Beschlussfassung {\"u}ber Satzungs{\"a}nderungen, Ausschluss von
                Mitgliedern und die Aufl{\"o}sung des Vereins.
     }
   \item Die ordentliche Mitgliederversammlung findet einmal j{\"a}hrlich statt.
   \item Eine au{\ss}erordentliche Mitgliederversammlung wird anberaumt
     \items{
       \it[(a)] auf Beschluss des Vorstands oder der Mitgliederversammlung;
       \it[(b)] auf Verlangen der Rechnungspr{\"u}fer oder auf schriftlichen Antrag eines
                Zehntels der Mitglieder; in diesem Fall ist die Versammlung innerhalb von
                vier Wochen nach Antragseingang durchzuf{\"u}hren.
     }
   \item Zu jeder Mitgliederversammlung sind alle Mitglieder mindestens zwei Wochen vor
       dem Termin schriftlich oder per E-Mail einzuladen.
       Die Einladung erfolgt durch ein Mitglied des Vorstands.
       In dringenden F{\"a}llen kann die Ladefrist unter Angabe des Dringlichkeitsgrundes
       auf eine Woche verk{\"u}rzt werden.
   \item
       Die Einladung der Mitgliederversammlung erfolgt unter Angabe der Tagesordnung.
       Beschlussfassung ist nur m{\"o}glich {\"u}ber Gegenst{\"a}nde, die in der Tagesordnung
       genannt sind. Davon abweichend ist ein Beschluss zur Einberufung einer
       au{\ss}erordentlichen Mitgliederversammlung jederzeit m{\"o}glich.
       Zur Beschlussfassung {\"u}ber eine {\"A}nderung der Satzung muss die vorgeschlagene
       Neufassung der Einladung im Wortlaut beif{\"u}gt werden.
    \item Antr{\"a}ge zur Tagesordnung
      \it[(a)] schriftliche Antr{\"a}ge eines Zehntels der Mitglieder werden in die Tagesordnung
               aufgenommen, sofern dies unter Einhaltung der Ladefrist m{\"o}glich ist;
      \it[(b)] ohne R\"ucksicht auf die Ladefrist wird die 
       Auf Verlangen des Vorstandes, auf Antrag eine Zehntels der Mitglieder  oder eines Drittels
       der erschienenen Mitglieder wird die Tagesordnung erg{\"a}nzt um Gegenst\"ande, {\"u}ber die kein Beschluss gefasst werden
       soll oder die lediglich die Einberufung einer ausserordentlichen Mitgliederversammlung
       oder die Versammlungsleitung betreffen.
   \item
       Die Mitgliederversammlung ist {\"o}ffentlich und wird vom Vorsitzenden des Vereins
       geleitet, sofern die Mitgliederversammlung nichts anderes beschliesst.
       Der Versammlungsleiter bestimmt im Einvernehmen mit der Mitgliederversammlung ein
       anwesendes Mitglied zum Schriftf{\"u}hrer.
   \item 
     Beschlussf{\"a}higkeit
       items{
         \it[(a)] Jede ordnungsgem{\"a}ss einberufene Mitgliederversammlung ist ohne R{\"u}cksicht auf
         die Anzahl der erschienenen Mitglieder beschlussf{\"a}hig.
         \it[(b)] Davon abweichend erfordern Beschl{\"u}sse {\"u}ber Satzungs{\"a}nderungen oder
         die Aufl{\"o}sung des Vereins die
         Anwesendheit der H{\"a}lfte aller Mitglieder. Liegt insoweit Beschlussunf{\"a}higkeit
         vor, kann die Mitgliederversammlung die Einberufung eine ausserordentlichen
         Mitgliederversammlung beschliessen, die ohne R{\"u}cksicht auf die Zahl der Erschienenen zum
         Beschluss {\"u}ber die neue Satzung berechtigt ist.
       }
   \item
     Die Mitgliederversammlung beschlie{\ss}t
     \items{
       \it[(a)] {\"u}ber die Auf{\"o}sung des Vereins mit  einer Mehrheit von 3/4 der anwesenden Mitglieder;
       \it[(b)] {\"u}ber Satzungs{\"a}nderungen und Ausschluss von Mitgliedern mit einer Mehrheit von
                2/3 der anwesenden Mitglieder;
       \it[(c)] in allen anderen F{\"a}llen mit einfacher Mehrheit.
     }
   \item Den Vorsitz in der Mitgliederversammlung f{\"u}hrt der Vereinsvorsitzende, in
       dessen Verhinderung sein Stellvertreter. Wenn auch dieser verhindert ist,
       w{\"a}hlt die Mitgliederversammlung aus ihrer Mitte einen Versammlungsleiter.
       Als Schriftf{\"u}hrer fungiert das zum Schriftf{\"u}hrer gew{\"a}hlte Vorstandsmitglied,
       bei dessen Verhinderung bestimmt der Versammlungsleiter ein anwesendes
       Vereinmitglied als Schriftf{\"u}hrer.
   \item Die Wahlen und die Beschlussfassungen in der Mitgliederversammlung
       erfolgen in der Regel mit einfacher Mehrheit der abgegebenen g{\"u}ltigen
       Stimmen. Die Entlastung des Vorstandes bedarf jedoch einer qualifizierten
       Mehrheit von zwei Dritteln der abgegebenen g{\"u}ltigen Stimmen. Ebenso
       bed{\"u}rfen Beschl{\"u}sse {\"u}ber den Ausschluss von Mitgliedern sowie
       Beschl{\"u}sse, mit denen die Satzung des Vereins ge{\"a}ndert werden soll, einer
       qualifizierten Mehrheit.
   \item Die Abstimmungsart bestimmt in der Regel der Versammlungsleiter. Eine
       schriftliche Abstimmung muss erfolgen, wenn ein Drittel der erschienenen
       Mitglieder dies beantragt.
   \item Im Rahmen der ordentlichen Mitgliederversammlung erfolgt der Bericht des
        Kassenwartes sowie der Kassenpr{\"u}fer sowie die Entlastung des Vorstandes.
   \item Zwei Rechnungspr{\"u}fer werden von der Mitgliederversammlung auf die Dauer
        von einem Jahr gew{\"a}hlt. Wiederwahl ist m{\"o}glich. Die Rechnungspr{\"u}fer
        d{\"u}rfen nicht dem Vorstand angeh{\"o}ren.
   \item Die gefassten Beschl{\"u}sse werden schriftlich niedergelegt und vom
        Versammlungsleiter und vom Schriftf{\"u}hrer unterzeichnet. Das Protokoll hat
        Ort und Zeit der Versammlung sowie die Abstimmungsergebnisse zu
        enthalten.
}


\paragraf8{Vereinsmittel}

\items{
   \item F{\"u}r die Erf{\"u}llung der satzungsm{\"a}{\ss}igen Zwecke notwendige materielle Mittel
        k{\"o}nnen aus Beitr{\"a}gen/Umlagen, Spenden, Zusch{\"u}sse und sonstigen
        Zuwendungen erlangt werden. Die Mitgliederversammlung kann eine
        Beitragsordnung erlassen, die H{\"o}he und Turnus von Mitgliedsbeitr{\"a}gen
        regelt.
   \item Der Betrag, {\"u}ber den der Vorstand ohne Beschluss der
        Mitgliederversammlung verf{\"u}gen kann, wird durch die Mitgliederversammlung
        festgelegt.
}

\paragraf{10}{Hochschulgruppe}

\items{
  \item
    Mitglieder des Vereins, die Mitglieder Brandenburger Hochschulen oder
    Fachhochschulen sind, bilden die \glqq upLUG Hochschulgruppe\grqq (HG).
  \item
    Mitgliederversammlungen des Vereins sind zugleich Mitgliederversammlungen der HG.
    {\"U}ber Angelegenheiten, die nur die HG betreffen, sind nur die Mitglieder der HG
    stimmberechtigt.
  \item
    Diejenigen Mitglieder des Vereinsvorstandes, die zugleich
    Mitglieder der HG sind, bilden den Vorstand der HG, sofern die HG nichts anderes beschliesst.
}

\paragraf{11}{Aufl{\"o}sung des Vereins}

\items{
   \item
     Sofern die Mitgliederversammlung nichts anderes beschlie{\ss}t, wird im Falle der Auf{\"o}sung
     der letzte Vereinsvorstand mit der Liquidation beauftragt. Je zwei Liquidatoren sind
     gemeinsam vertretungsberechtigt.
   \item
     Das nach Begleichung aller Verbindlichkeiten verbleibende Vereinsverm{\"o}gen
     f{\"a}llt an die Studierendenschaft der Universit{\"a}t Potsdam vertreten durch
     den Allgemeinen Studierendenausschuss der Universit{\"a}t Potsdam. Die genannte
     Institution hat das Verm{\"o}gen unmittelbar und ausschlie{\ss}lich f{\"u}r
     gemeinn{\"u}tzige Zwecke zu verwenden.
}

\par\bigskip
Potsdam, den 22.\,07.\,2012

\end{document}

